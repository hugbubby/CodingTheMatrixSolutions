\documentclass{article}
\usepackage{amsmath}
\usepackage{amssymb}

\title{Coding the Matrix --- Written Questions}
\author{Dean F. Valentine Jr.}

\begin{document}
    \maketitle
    Here we will have the written answers to selected problems that do not
    require code or drawings to solve.
    \section{The Field}
    \addtocounter{subsection}{5}
    \subsection{Review}
    \begin{enumerate}
        \item The complex numbers, the reals, and the integers.
        \item z.real $-$ z.imag, and the formula for the absolute value of a complex number is $z * z_{c}$
        \item Adding the real and imaginary components seperately.
        \item Putting them in an equation and using distributive property.
        \item Adding two complex numbers together.
        \item Multiplying a real number by a complex number.
        \item Multiplying by -1.
        \item Multiplying by $e^{\frac{{\pi}i}{2}}$.
        \item Adding the two bits and then applying modulo 2.
        \item Setting the result to 0 if one of the bits is 0 and 1 otherwise.
    \end{enumerate}
    \subsection{Problems}
    \begin{enumerate}
        \addtocounter{enumi}{9}
        \item 
            \begin{enumerate}
                \item $5 + 3i$
                \item $i$
                \item $-1 + 0.001i$
                \item $0.001 + 9i$
            \end{enumerate}
        \item
            \begin{enumerate}
                \item $e^{3i}$
                \item $e^{(\frac{11\pi}{12})i}$
                \item $e^{(\frac{5\pi}{12})i}$
            \end{enumerate}
        \item
            \begin{enumerate}
                \item $a = (2)(e^{(\frac{\pi}{2})i})$, $b = 1 + 1i$
                \item Not possible to scale the real part by two and imaginary
                    part by three in only one multiplication.
            \end{enumerate}
        \item
            \begin{enumerate}
                \item $1 + 1 + 1 + 0 = (1 + 1) + (1 + 0) = (0) + (1) = 1$
                \item $0$
                \item $0$
            \end{enumerate}
    \end{enumerate}
    \section{The Vector}
    \addtocounter{subsection}{5}
    \subsection{Combining Vector Addition and Scalar Multiplication}
    \begin{enumerate}
        \item $[3,4]$, and the translation vector is $[2,3]$. 
        \item $\{\alpha [5,-1] + [1,4] : \alpha \in \mathbb{R}, 0 \leq \alpha \leq 1\}$
        \addtocounter{enumi}{3}
        \item Same approach as in 2.5.5. For each element $k$ of the domain
            $D$, entry $k$ of $(\alpha + \beta)u$ is $(\alpha + \beta)u[k]$. By
            the distributive law for fields, this is equal to $\alpha u[k] +
            \beta u[k]$, which is the $k$th element of $\alpha u + \beta u$.
    \end{enumerate}
    \stepcounter{subsection}
    \subsection{Vectors over $GF(2)$}
    \begin{enumerate}
        \addtocounter{enumi}{2}
        \item Have two n-bit keys chosen randomly and uniformly over $GF(2)$ this
            time, $v_a$ and $v_b$, given to Alice and Bob. Give a third TA the key
            $v_c := v - v_a - v_b$.
    \end{enumerate}
    \subsection{Dot Product}
    \begin{enumerate}
        \addtocounter{enumi}{22}
        \item Suppose $u = [u_1, u_2$ \ldots $u_n]$ and $v = [v_1, v_2$ \ldots $v_n]$. \\
            $(\alpha u) * v$ \\
            $= (\alpha u_1)(v_1) + (\alpha u_2)(v_2)$ \ldots $(\alpha u_n)(v_n)$ \\
            $= \alpha (u_1)(v_1) + \alpha (u_2)(v_2)$ \ldots $\alpha (u_n)(v_n)$ \\
            $= \alpha ((u_1)(v_1) + (u_2)(v_2)$ \ldots $(u_n)(v_n))$ \\
            $= \alpha (u * v)$
        \item $\alpha = 2, u = [2,2], v = [2,2]$ \\
            $(2 * [2,2]) * (2 * [2,2]) = [4,4] * [4,4] = 32$ \\
            $2 * ([2,2] * [2,2]) = 2 * (4 + 4) = 2 * 8 = 16$
        \stepcounter{enumi}
        \item $u,v,w,x = [2,2]$ \\
            $(u + v) * (w + x) = [4,4] * [4,4] = 32$ \\
            $(u * v) + (w * x) = 8 + 8 = 16$
        \addtocounter{enumi}{2}
        \item For the first number, sum the last three challenges, $101010 + 111011 + 001100 = 011101$. \\
            The sum of their responses is 0. \\
            For the second number, sum the first, third, and last challenge,
            $110011 + 111011 + 001100 = 000100$. \\
            The sum of their responses is also 0.
    \end{enumerate}
    \stepcounter{subsection}
    \subsection{Solving a triangular system of linear equations}
    \begin{enumerate}
        \addtocounter{enumi}{2}
        \item $x_3 = -\frac{6}{5}$ \\
            $x_2 = \frac{4 - 4x_3}{2} = 2 - 2x_3 = 2 - 2(-\frac{6}{5}) = 2 + \frac{12}{5} = \frac{22}{5}$ \\
            $x_1 = 7 + 3x_2 + 2x_3 = 7 + 3(-\frac{6}{5}) + 2(\frac{22}{5}) = 7 - \frac{18}{5} + \frac{44}{5}$\\
            $= 7 + \frac{26}{5} = \frac{61}{5}$
    \end{enumerate}
    \stepcounter{subsection}
    \subsection{Review Questions}
    \begin{itemize}
        \item Vector addition is, as the name suggests, the summation of two or
            more vectors.  If you consider, as the book does, vectors to be
            special types of functions, then vector addition is the act of
            creating a new function equal to the sum of two ``vector functions''.
        \item Points in an $X$-D space can be modeled as arrows from the origin
            in that $X$-D space.  Similarly, the sum of vectors that can be represented as
            points can be interpreted as ``summing'' the arrows, or laying their displacements
            from the origin on top of each other.
        \item Scalar-Vector multiplication is the act of multiplying a
            ``scalar'', that's within a field, and a vector together, in the
            manner that the above definition of vector addition would imply.
        \item $(\alpha + \beta)u = \alpha u + \beta u$
        \item $\alpha (u + v) = \alpha u + \alpha v$
        \item The set of all possible solutions to the equation $\alpha [x,y]$,
            which is analogous to the line connecting and moving beyond the
            points $(0,0)$ and $(x,y)$.
        \item The similar set $\{\alpha [w,x] + \beta [y,z] : \alpha, \beta \in
            \mathbb{F}, \alpha + \beta = 1\}$, where $\mathbb{F}$ is your field of choice.
        \item The sum of the product of the corresponding entries of two
            vectors. We defined vectors as functions earlier and so I don't
            know if this is a sane way to go about talking about dot products,
            but here is the book's definition. Two vectors must have the same
            domain for this dot product definition to make sense.
        \item $(\alpha u) * v = \alpha (u * v)$
        \item $(u + v) * w = u * w + v * w$
        \item An equation of the form $a * x = \beta$, where $a$ is a static
            vector, $b$ is a vector variable, and $\beta$ is a scalar.
        \item A set of linear equations.
        \item A linear system in upper-triangular form.
        \item Solving for the variable at the bottom, and progressing orderly
            to the top, solving for one variable each equation, until you have
            reached the top of the triangle and solved for all $x_k$.
    \end{itemize}
    \section{Problems}
    \begin{enumerate}
        \item $[-1, 7]$, \\
            $[-1, -1]$, and \\
            $[-3, 1]$.
        \item $[1,0,6]$, \\
            $[3,-2,-4]$, \\
            $[5,-3,9]$, \\
            $[0,1,7]$
        \item $[one,0,0]$ \\
            $[0,one,one]$
        \item 
            \begin{enumerate}
                \item $c + d + e$
                \item $b + c + d + e$
            \end{enumerate}
        \item 
            \begin{enumerate}
                \item $c + d$
                \item Could not find one.
            \end{enumerate}
        \item 
            \begin{itemize}
                \item 1011
                \item 1101
                \item 1000
                \item 1011 + 1111 = 0100 \\
                    0100 * 1100 = 0 + 1 + 0 + 0 = 1 \\
                    1101 + 1111 = 0010 \\
                    1010 + 0010 = 0 + 0 + 1 + 0 = 1 \\
                    1000 + 1111 = 0111 \\
                    1111 * 0111 = 0 + 1 + 1 + 1 = 1 + 0 = 1
            \end{itemize}
        \item 
            \begin{enumerate} 
                \item $v_1 = [2,3,4,3]$
                \item $v_2 = [1,-5,2,0]$
                \item $v_3 = [4,1,-1,-1]$
            \end{enumerate}
            \stepcounter{enumi}
        \item 
            \begin{enumerate}
                \item 5
                \item 6
                \item 16
                \item $-1$
            \end{enumerate}
    \end{enumerate}
\end{document}
