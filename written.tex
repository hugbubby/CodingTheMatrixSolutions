\documentclass{article}
\usepackage{amsmath}
\usepackage{amssymb}
\usepackage{enumerate}

\title{Coding the Matrix --- Written Questions}
\author{Dean F. Valentine Jr.}

\begin{document}
    \maketitle
    Here we will have the written answers to selected problems that do not
    require code or drawings to solve.
    \section{The Field}
    \addtocounter{subsection}{5}
    \subsection{Review}
    \begin{enumerate}
        \item The complex numbers, the reals, and the integers.
        \item z.real $-$ z.imag, and the formula for the absolute value of a complex number is $z * z_{c}$
        \item Adding the real and imaginary components seperately.
        \item Putting them in an equation and using distributive property.
        \item Adding two complex numbers together.
        \item Multiplying a real number by a complex number.
        \item Multiplying by -1.
        \item Multiplying by $e^{\frac{{\pi}i}{2}}$.
        \item Adding the two bits and then applying modulo 2.
        \item Setting the result to 0 if one of the bits is 0 and 1 otherwise.
    \end{enumerate}
    \subsection{Problems}
    \begin{enumerate}
        \addtocounter{enumi}{9}
        \item 
            \begin{enumerate}
                \item $5 + 3i$
                \item $i$
                \item $-1 + 0.001i$
                \item $0.001 + 9i$
            \end{enumerate}
        \item
            \begin{enumerate}
                \item $e^{3i}$
                \item $e^{(\frac{11\pi}{12})i}$
                \item $e^{(\frac{5\pi}{12})i}$
            \end{enumerate}
        \item
            \begin{enumerate}
                \item $a = (2)(e^{(\frac{\pi}{2})i})$, $b = 1 + 1i$
                \item Not possible to scale the real part by two and imaginary
                    part by three in only one multiplication.
            \end{enumerate}
        \item
            \begin{enumerate}
                \item $1 + 1 + 1 + 0 = (1 + 1) + (1 + 0) = (0) + (1) = 1$
                \item $0$
                \item $0$
            \end{enumerate}
    \end{enumerate}
    \section{The Vector}
    \addtocounter{subsection}{5}
    \subsection{Combining Vector Addition and Scalar Multiplication}
    \begin{enumerate}
        \item $[3,4]$, and the translation vector is $[2,3]$. 
        \item $\{\alpha [5,-1] + [1,4] : \alpha \in \mathbb{R}, 0 \leq \alpha \leq 1\}$
        \addtocounter{enumi}{3}
        \item Same approach as in 2.5.5. For each element $k$ of the domain
            $D$, entry $k$ of $(\alpha + \beta)u$ is $(\alpha + \beta)u[k]$. By
            the distributive law for fields, this is equal to $\alpha u[k] +
            \beta u[k]$, which is the $k$th element of $\alpha u + \beta u$.
    \end{enumerate}
    \stepcounter{subsection}
    \subsection{Vectors over $GF(2)$}
    \begin{enumerate}
        \addtocounter{enumi}{2}
        \item Have two n-bit keys chosen randomly and uniformly over $GF(2)$ this
            time, $v_a$ and $v_b$, given to Alice and Bob. Give a third TA the key
            $v_c := v - v_a - v_b$.
    \end{enumerate}
    \subsection{Dot Product}
    \begin{enumerate}
        \addtocounter{enumi}{22}
        \item Suppose $u = [u_1, u_2$ \ldots $u_n]$ and $v = [v_1, v_2$ \ldots $v_n]$. \\
            $(\alpha u) * v$ \\
            $= (\alpha u_1)(v_1) + (\alpha u_2)(v_2)$ \ldots $(\alpha u_n)(v_n)$ \\
            $= \alpha (u_1)(v_1) + \alpha (u_2)(v_2)$ \ldots $\alpha (u_n)(v_n)$ \\
            $= \alpha ((u_1)(v_1) + (u_2)(v_2)$ \ldots $(u_n)(v_n))$ \\
            $= \alpha (u * v)$
        \item $\alpha = 2, u = [2,2], v = [2,2]$ \\
            $(2 * [2,2]) * (2 * [2,2]) = [4,4] * [4,4] = 32$ \\
            $2 * ([2,2] * [2,2]) = 2 * (4 + 4) = 2 * 8 = 16$
        \stepcounter{enumi}
        \item $u,v,w,x = [2,2]$ \\
            $(u + v) * (w + x) = [4,4] * [4,4] = 32$ \\
            $(u * v) + (w * x) = 8 + 8 = 16$
        \addtocounter{enumi}{2}
        \item For the first number, sum the last three challenges, $101010 + 111011 + 001100 = 011101$. \\
            The sum of their responses is 0. \\
            For the second number, sum the first, third, and last challenge,
            $110011 + 111011 + 001100 = 000100$. \\
            The sum of their responses is also 0.
    \end{enumerate}
    \stepcounter{subsection}
    \subsection{Solving a triangular system of linear equations}
    \begin{enumerate}
        \addtocounter{enumi}{2}
        \item $x_3 = -\frac{6}{5}$ \\
            $x_2 = \frac{4 - 4x_3}{2} = 2 - 2x_3 = 2 - 2(-\frac{6}{5}) = 2 + \frac{12}{5} = \frac{22}{5}$ \\
            $x_1 = 7 + 3x_2 + 2x_3 = 7 + 3(-\frac{6}{5}) + 2(\frac{22}{5}) = 7 - \frac{18}{5} + \frac{44}{5}$\\
            $= 7 + \frac{26}{5} = \frac{61}{5}$
    \end{enumerate}
    \stepcounter{subsection}
    \subsection{Review Questions}
    \begin{itemize}
        \item Vector addition is, as the name suggests, the summation of two or
            more vectors.  If you consider, as the book does, vectors to be
            special types of functions, then vector addition is the act of
            creating a new function equal to the sum of two ``vector functions''.
        \item Points in an $X$-D space can be modeled as arrows from the origin
            in that $X$-D space.  Similarly, the sum of vectors that can be represented as
            points can be interpreted as ``summing'' the arrows, or laying their displacements
            from the origin on top of each other.
        \item Scalar-Vector multiplication is the act of multiplying a
            ``scalar'', that's within a field, and a vector together, in the
            manner that the above definition of vector addition would imply.
        \item $(\alpha + \beta)u = \alpha u + \beta u$
        \item $\alpha (u + v) = \alpha u + \alpha v$
        \item The set of all possible solutions to the equation $\alpha [x,y]$,
            which is analogous to the line connecting and moving beyond the
            points $(0,0)$ and $(x,y)$.
        \item The similar set $\{\alpha [w,x] + \beta [y,z] : \alpha, \beta \in
            \mathbb{F}, \alpha + \beta = 1\}$, where $\mathbb{F}$ is your field of choice.
        \item The sum of the product of the corresponding entries of two
            vectors. We defined vectors as functions earlier and so I don't
            know if this is a sane way to go about talking about dot products,
            but here is the book's definition. Two vectors must have the same
            domain for this dot product definition to make sense.
        \item $(\alpha u) * v = \alpha (u * v)$
        \item $(u + v) * w = u * w + v * w$
        \item An equation of the form $a * x = \beta$, where $a$ is a static
            vector, $b$ is a vector variable, and $\beta$ is a scalar.
        \item A set of linear equations.
        \item A linear system in upper-triangular form.
        \item Solving for the variable at the bottom, and progressing orderly
            to the top, solving for one variable each equation, until you have
            reached the top of the triangle and solved for all $x_k$.
    \end{itemize}
    \subsection{Problems}
    \begin{enumerate}
        \item $[-1, 7]$, \\
            $[-1, -1]$, and \\
            $[-3, 1]$.
        \item $[1,0,6]$, \\
            $[3,-2,-4]$, \\
            $[5,-3,9]$, \\
            $[0,1,7]$
        \item $[one,0,0]$ \\
            $[0,one,one]$
        \item 
            \begin{enumerate}
                \item $c + d + e$
                \item $b + c + d + e$
            \end{enumerate}
        \item 
            \begin{enumerate}
                \item $c + d$
                \item Could not find one.
            \end{enumerate}
        \item 
            \begin{itemize}
                \item 1011
                \item 1101
                \item 1000
                \item 1011 + 1111 = 0100 \\
                    0100 * 1100 = 0 + 1 + 0 + 0 = 1 \\
                    1101 + 1111 = 0010 \\
                    1010 + 0010 = 0 + 0 + 1 + 0 = 1 \\
                    1000 + 1111 = 0111 \\
                    1111 * 0111 = 0 + 1 + 1 + 1 = 1 + 0 = 1
            \end{itemize}
        \item 
            \begin{enumerate} 
                \item $v_1 = [2,3,4,3]$
                \item $v_2 = [1,-5,2,0]$
                \item $v_3 = [4,1,-1,-1]$
            \end{enumerate}
            \stepcounter{enumi}
        \item 
            \begin{enumerate}
                \item 5
                \item 6
                \item 16
                \item $-1$
            \end{enumerate}
    \end{enumerate}
    \section{The Vector Space}
    \stepcounter{subsection}
    \subsection{Span}
    \begin{enumerate}
        \addtocounter{enumi}{14}
        \item 
            \begin{enumerate}
                \item $[3,4] + (-2)[1,2] = [3,4] - [2,4] = [1,0]$ \\
                    $(-\frac{1}{2})[3,4] + (\frac{3}{2})[1,2] = [-\frac{3}{2},-2] + [\frac{3}{2},3] = [0,1]$ \\
                    It is possible to write each of the standard generators, so this is a generator.
                \item This is not a generator and it is impossible to write either of the standard generators.
                    We add $[0,2]$: \\
                    $(\frac{1}{2})[0,2] + (0)[1,1] + (0)[2,2] + (0)[3,3] = [0,1]$ \\
                    $(-\frac{1}{2})[0,2] + [1,1] + (0)[2,2] + (0)[3,3] = [1,0]$ \\
                    So this is now a generator.
                \item We have one of the standard generators already, $[0,1]$. \\
                    $[1,1] + (0)[1,-1] + (-1)[0,1] = [1,1] - [0,1] = [1,0]$, so this is a generator.
            \end{enumerate}
        \item Our new vector is $[0,0,1]$, one of the standard generators. \bigskip\\
            $(-\frac{4}{9})[1,1,1] + (\frac{10}{9})[0.4,1.3,-2.2] + (\frac{22}{9})[0,0,1] =$  \medskip\\
            $[-\frac{4}{9},-\frac{4}{9},-\frac{4}{9}] + [\frac{4}{9},\frac{13}{9},-\frac{22}{9}] + [0,0,\frac{26}{9}] =$ \medskip\\
            $[-\frac{4}{9},-\frac{4}{9},-\frac{4}{9}] + [\frac{4}{9},\frac{13}{9},\frac{4}{9}] = [0,\frac{9}{9},0] = [0,1,0]$ is \#2, and: \bigskip\\
            $[1,1,1] + (-1)[0,1,0] + (-1)[0,0,1] = [1,0,0]$ is \#3.
        \item 
    \end{enumerate}
    \addtocounter{subsection}{4}
    \subsection{Review Questions}
    \begin{itemize}
        \item A linear combination is a solution to an equation of the form
            $\alpha{}_1(u_1) + $ \ldots $ + \alpha{}_n(u_n)$, where $\alpha{}_1 \ldots \alpha{}_n$ is
            any given set of scalars and $u_1 \ldots u_n$ are a particular assortment of
            vectors. We say a vector $x$ is a \textit{linear combination of}
            the vectors $u_1 \ldots u_n$ if there exists such a sum, with
            those vectors, that equals $x$.
        \item Coefficients are multiplicative factors in a mathematical expression.
        \item The set of all linear combinations of those vectors.
        \item The standard generators for a field of dimension $n$ are the $n$ vectors 
            that are each a list of zeros excepting a single one in different and distinct
            positions, such that their span is equal to the field.
        \item $\{Span \{[1,1]\}\}$ and $\{Span \{[1,1,1], [1,2,3]\}\}$
        \item A linear equation the ``right hand side'' of which is equal to zero.
        \item A concurrent set of homogenous linear equations.
        \item The span of a set of vectors, or the solution set to a homogenous linear system.
        \item A vector space ``over $\mathbb{F}$'', where $\mathbb{F}$ is some
            field, is a set of objects such that the following three
            properties hold:
            \begin{enumerate}
                \item The set contains the zero vector.
                \item The sum of two members of the set is also inside the set (It is \textit{``closed under vector addition''}).
                \item The product of a scalar in $\mathbb{F}$ and a member of the set is also inside the set (It is \textit{``closed under scalar multiplication''}).
            \end{enumerate}
        \item A subspace is a subset of a vector space that is also, itself, a vector space.
        \item An affine combination is a linear combination of some set of vectors where the scalar coefficients all add up to one.
        \item An affine hull of vectors is the set of all affine combinations of some sert of vectors.
        \item An affine space is the set of all possible solutions to the sum of a specified vector and all the vectors in some vector space.
        \item Two kinds of representations are an affine hull, and a translation of flat containing the origin.
    \end{itemize}
    \subsection{Problems}
    \begin{enumerate}
        \addtocounter{enumi}{3}
        \item 
            \begin{itemize}
                \item If $v_1 = [x_1,y_1,z_1] = [1,0,a]$, $ax_1 + by_1 = a(1) + b(0) = a
                    = z_1$, for all choice of a. Since a = z, for any coefficient
                    $\alpha$, $\alpha a = \alpha z$, and so $\forall v \in Span{v_1}: z = ax + by$.
                \item Similarly, if $v_2 = [x_2,y_2,z_2] = [0,1,b] \implies ax_2 + by_2 = a(0) + b(1) = b = z_2$.
                     Since a = z, for any coefficient $\beta$, $\beta a = \beta z$, and so $\forall v \in Span{v_2}: z = ax + by$.
                 \item $Span\{v_1, v_2\}$ is equal to $\{u + w, u \in Span\{v_1\}, w \in Span\{v_2\}\}$. Since $\forall u, w: z = ax+by$,
                     $\forall (u+w):  z_1 + z_2 = z = a(x_1 + x_2) + b(y_1 + y_2) = ax + by$. So $\forall v \in Span{v_1, v_2}: z = ax + by$.
                 \item Insert other backwards proof here.
            \end{itemize}
        \item This proof is nearly identical to the one above.
        \item 
            \begin{enumerate}
                \item $\{\alpha [1,3], \alpha \in \mathbb{R}, 0 < \alpha < 1\}$
                \item $\{\alpha [2,2,2] + \beta [2,2,0], \alpha,\beta \in mathbb{R}, 0 < \alpha, \beta < 1, \alpha + \beta = 1\}$
            \end{enumerate}
        \item
            \begin{enumerate}
                \item It does not contain the zero vector, so it is not a vector space.
                \item 
                    \begin{enumerate}
                        \item It contains the zero vector $[0,0,0]$
                        \item It is closed under vector addition, because $x_1 + y_1 + z_1 = 0 \land x_2 + y_2 + z_2 = 0 \implies (x_1 + x_2) + (y_1 + y_2) + (z_1 + z_2) = (x_1 + y_1 + z_1) + (x_2 + y_2 + z_2) = 0 + 0 = 0$
                        \item It is also cloed under scalar multiplication, because $x_1 + y_1 + z_1 = 0 \implies \alpha (x_1 + y_1 + z_1) = \alpha (0) = 0$
                    \end{enumerate}
                    So, this is a vector space.
                \item 
                    \begin{enumerate}
                        \item It contains the zero vector $[0,0,0,0,0]$
                        \item It is closed under vector addition, because ${x_2}_1 = 0 \land {x_2}_2 = 0 \implies ({x_2}_1 + {x_2}_2) = 0$, and likewise for ${x_5}_1$ and ${x_5}_2$
                        \item It is closed under scalar multiplication, because $x_2 = 0 \implies \alpha (x_2) = \alpha (0) = 0$, and likewise for $\alpha (x_5)$
                    \end{enumerate}
                    So, this is a vector space.
            \end{enumerate}
        \item 
            \begin{enumerate}
                \item As far as I can tell, yes, but I have not proved it.
                \item No, firstly because it does not contain the zero vector, $[0,0,0,0,0]$. This contains an even number of ones, specifically $0$.
            \end{enumerate}
    \end{enumerate}
    \section{The Matrix}
    \setcounter{subsection}{5}
    \subsection{Matrix-Vector Multiplication in terms of dot products}
    \begin{enumerate}
        \setcounter{enumi}{13}
        \item 
            \begin{enumerate}
                \item Entry $r$ of the left hand side equals the dot product of row $r$ of $M$ with $(u+v)$
                \item The first entry $r$ of the right hand side equals the dot product of row $r$ of $M$ with $(u)$
                \item The second entry $r$ of the right hand side equals the dot product of row $r$ of $M$ with $(v)$
                \item The entry $r$ of the sum of those last two vectors equals the sum of those two dot products.
                \item $r * (u+v) = r*u + r*v$, becuse of the distributive property of the dot product
            \end{enumerate}
    \end{enumerate}
    \subsection{Null space}
    \begin{enumerate}
        \setcounter{enumi}{2}
        \item 
            \begin{enumerate}
                \item $[-1, 0, 1]$
                \item $[0, -1, 1]$
                \item $[0, 1, 0]$
            \end{enumerate}
    \end{enumerate}
    \stepcounter{subsection}
    \subsection{The Matrix meets the function}
    \begin{enumerate}
        \stepcounter{enumi}
        \item 
            \begin{enumerate}
                \item $\mathbb{R}^{\{a,b\}}$
                \item $\mathbb{R}^{\{\#,@,?\}}$
                \item $\{a: 0, b: 0\}$
            \end{enumerate}
    \end{enumerate}
    \subsection{Linear Functions}
    \begin{enumerate}
        \setcounter{enumi}{6}
        \item No, because $f([1,1]) + f([1,1]) = [2,2,2]$ whereas $f([1,1] + [1,1]) = [2,2,1]$, and it defies property two.
        \item 
            \begin{enumerate}
                \item $h([x,y]) = [x,(-1)*y]$.
                \item Yes. Multiplication of one part of the vector by -1 satisfies both properties one and two.
            \end{enumerate}
        \item Example 4.9.6. The image of $[1,0]$ is $[2,2]$, but the image of
            $[2,0]$ and $[3,0]$ is $[3,2]$ and $[4,2]$, respectively, according
            to the definition of the function.  Since $f([1,0] + [2,0]) = [4,2]
            \neq f([1,0]) + f([2,0]) = [5,4]$, this is not a linear function.
            \setcounter{enumi}{12}
        \item 
            \begin{enumerate}
                \item If $f$ is a linear map and $f(x) = 0$, then $\forall \alpha \in
                    \mathbb{F}: f(\alpha x) = \alpha f(x) = 0$ by property one of linear maps.
                    As a result, $\forall \alpha \in \mathbb{F}, x \in$ Ker $f: \alpha x \ in$ Ker $f$.
                \item Similarly, if $f$ is a linear map and $f(x) = 0$, then
                    $f(y) = 0 \implies f(x + y) = f(x) + f(y) = 0 + 0 = 0$ by
                    property two of linear maps. This means that
                    $\forall x,y \in$ Ker $f: x+y \in$ Ker $f$.
                \item It contains the zero vector, because of Lemma 4.10.10
            \end{enumerate}
    \end{enumerate}
    \setcounter{subsection}{12}
    \subsection{From function inverse to matrix inverse}
    \begin{enumerate}
        \stepcounter{enumi}
        \item Let $x = g(y)$.
            $g(\alpha y) = g(\alpha f(x))$ \\
            $g(\alpha f(x)) = g(f(\alpha x))$ \\
            $g(f(\alpha x)) = \alpha x$, so: \\
            $g(\alpha y) = \alpha g(y) = \alpha x$
    \end{enumerate}
    \setcounter{subsection}{15}
    \subsection{Review Questions}
    \begin{itemize}
        \item The transpose of an R x C matrix is the C x R matrix for which the position of all entries at r,c have been flipped to appear at c,r
        \item The sparsity of a matrix is the extent to which values tend to be
            zero, which, when combined with the standard that all row/label
            pairs that do not appear are assumed zero, can speed up many matrix
            operations and make matrices easier to store in memory.
        \item The linear combination definition of matrix-vector multiplication
            is a linear combination of the matrix's columns, the coefficients
            of which are the column-label-matching entries of the vector.
        \item The linear combination definition of vector-matrix multiplication
            is a linear combination of the matrix's rows, the coefficients of
            which are the row-label-matching entries of the vector.
        \item The dot product definition of matrix-vector multiplication
            is a series of dot products between the columns of the matrix and
            the vector, each consisting of an entry of the resulting vector.
        \item The dot product definition of vector-matrix multiplication
            is a series of dot products between the rows of the matrix and the
            vector, each consisting of an entry of the resulting vector.
        \item An identity matrix I for a given matrix M is the simplest
            diagonal matrix such that IM = M.
        \item An upper triangular matrix is a square matrix such that, within each column i,
            all rows greater than i contain the entry zero. 
        \item A matrix that only has nonzero entries in the locations where the row and column 
            have the same index.
        \item A linear function is a function whose domain is a vector space, whose codomain is
            a vector space, and which satisfies the following two properties:
            \begin{enumerate}
                \item $f(x + y) = f(x) + f(y)$
                \item $f(\alpha x) = \alpha f(x)$
            \end{enumerate}
            where 0, x, and y are vectors, and $\alpha$ is a scalar.
        \item It can be represented as the product of a m x n matrix
            M and a vector, namely, $f(x) = M * x$, or the transpose
            of that matrix, an n $\times$ m one we can name $M_T$ and thus define the
            function by $f(x) = x * M_T$.
        \item The kernel of a linear function is the set of all vectors in
            its co-domain that produce the zero vector. The image of a linear
            function is the set of all outputs of its domain.
        \item The null space of a matrix A, analogous to the kernel of a linear function,
            is the set of vectors such that $A * v$ equals the zero vector. The
            row space of A is the vector space spanned by the rows of A, and
            the column space is the vector space spanned by the columns of A.
        \item The matrix where, for each row-label r of A, row r of AB = (row r of A) * B. 
        \item The matrix where, for each column-label c of A, column c of AB = A * (column c of B).
        \item The matrix where entry entry rc of AB is the dot product of row r of A with column c of B.
        \item The associativity property of matrix-matrix multiplication says
            that the order in which multiplication is performed does not change
            the results of the equation.
        \item By defining the vector as a single-column or single-row matrix,
            you can generalize matrix-matrix multiplication to include
            matrix-vector multiplication.
        \item The outer-product of two vectors $u, v$, represented as single
            column matrices, is the product $uv^T$.
        \item It can be represented as the product of two matrices $u,v$,
            $u$ being a $1 x n$ vector and $v$ being an $n \times 1$ vector.
        \item The inverse of a matrix $M$ is the matrix $N$ whose linear map
            is the functional inverse of the linear map of $M$.
        \item If the matrix $N$ is to be an inverse of the matrix $M$,
            $NM$ must be a valid product and $NM$ must equal the identity
            matrix for $M$.
    \end{itemize}
    \subsection{Problems}
    \begin{enumerate}
        \item
            \begin{enumerate}
                \item $[1, 0]$
                \item $[0, 4.44]$
                \item $[14, 20, 26]$
            \end{enumerate}
        \item 
            $
            \begin{bmatrix}
                0 & 1 \\
                1 & 0
            \end{bmatrix}
            $
        \item 
            $
            \begin{bmatrix}
                1 & 0 & 1 \\
                0 & 1 & 0 \\
                1 & 0 & 0
            \end{bmatrix}
            $
        \item 
            $
            \begin{bmatrix}
                2 & 0 & 0 \\
                0 & 4 & 0 \\
                0 & 0 & 3
            \end{bmatrix}
            $
        \item 
            \begin{enumerate}[(i)]
                \item Invalid
                \item Invalid
                \item 1 $\times$ 2
                \item 2 $\times$ 1
                \item Invalid
                \item 1 $\times$ 1
                \item 3 $\times$ 3
            \end{enumerate}
        \item 
            \begin{enumerate}[(i)]
                \item 
                    $
                    \begin{bmatrix}
                        8 & 13 \\
                        8 & 14
                    \end{bmatrix}
                    $
                \item 
                    $
                    \begin{bmatrix}
                        24 & 11 & 4 \\
                        3 & 3 & 0
                    \end{bmatrix}
                    $
                \item 
                    $
                    \begin{bmatrix}
                        3 & 13
                    \end{bmatrix}
                    $
                \item $[14]$
                \item 
                    $
                    \begin{bmatrix}
                        1 & 2 & 3 \\
                        2 & 4 & 6 \\
                        3 & 6 & 9
                    \end{bmatrix}
                    $
                \item 
                    $
                    \begin{bmatrix}
                        -2 & 4 \\
                        -1 & 1 \\
                        -1 & -3
                    \end{bmatrix}
                    $
            \end{enumerate}
        \item 
            \begin{enumerate}[(i)]
                \item 
                    $
                    AB = \begin{bmatrix}
                        5 & 2 & 0 & 1 \\
                        2 & 1 & -4 & 6 \\
                        2 & 3 & 0 & -4 \\
                        -2 & 3 & 4 & 0
                    \end{bmatrix} \\
                    BA = \begin{bmatrix}
                        1 & -4 & 6 & 2\\
                        3 & 0 & -4 & 2\\
                        3 & 4 & 0 & -2\\
                        2 & 0 & 1 & 5
                    \end{bmatrix}
                    $
                \item 
                    $
                    AB = \begin{bmatrix}
                        5 & 1 & 0 & 2\\
                        2 & 6 & -4 & 1\\
                        2 & -4 & 0 & 3\\
                        -2 & 0 & 4 & 3
                    \end{bmatrix} \\
                    BA = \begin{bmatrix}
                        3 & 4 & 0 & -2\\
                        3 & 0 & -4 & 2\\
                        1 & -4 & 6 & 2\\
                        2 & 0 & 1 & 5
                    \end{bmatrix}
                    $
                \item 
                    $
                    AB = \begin{bmatrix}
                        1 & 0 & 5 & 2 \\
                        6 & -4 & 2 & 1 \\
                        -4 & 0 & 2 & 3 \\
                        0 & 4 & -2 & 3
                    \end{bmatrix} \\
                    BA = \begin{bmatrix}
                        3 & 4 & 0 & -2 \\
                        1 & -4 & 6 & 2 \\
                        2 & 0 & 1 & 5 \\
                        3 & 0 & -4 & 2
                    \end{bmatrix}
                    $
            \end{enumerate}
        \item 
            \begin{enumerate}[(i)]
                \item 
                    $
                    \begin{bmatrix}
                        1 & b + a  \\
                        0 & 1
                    \end{bmatrix}
                    $
                \item 
                    $
                    A^2 =
                    \begin{bmatrix}
                        1 & 2 \\
                        0 & 1
                    \end{bmatrix} \\
                    A^3 =
                    \begin{bmatrix}
                        1 & 3 \\
                        0 & 1
                    \end{bmatrix} \\
                    A^n =
                    \begin{bmatrix}
                        1 & n \\
                        0 & 1
                    \end{bmatrix}
                    $
            \end{enumerate}
        \item 
            \begin{enumerate}
                \item 
                    $
                    AB =
                    \begin{bmatrix}
                        0 & 0 & 2 & 0 \\
                        0 & 0 & 5 & 0 \\
                        0 & 0 & 4 & 0 \\
                        0 & 0 & 6 & 0
                    \end{bmatrix} \\
                    BA =
                    \begin{bmatrix}
                        0 & 0 & 0 & 0 \\
                        4 & 4 & 4 & 0 \\
                        0 & 0 & 0 & 0 \\
                        0 & 0 & 0 & 0
                    \end{bmatrix}
                    $
                \item
                    $
                    AB =
                    \begin{bmatrix}
                        0 & 2 & -1 & 0 \\
                        0 & 5 & 3 & 0 \\
                        0 & 4 & 0 & 0 \\
                        0 & 6 & 5 & 0
                    \end{bmatrix} \\
                    BA =
                    \begin{bmatrix}
                        0 & 0 & 0 & 0 \\
                        1 & 5 & -2 & 3 \\
                        0 & 0 & 0 & 0 \\
                        4 & 4 & 4 & 0
                    \end{bmatrix}
                    $ 
                \item 
                    $
                    AB =
                    \begin{bmatrix}
                        6 & 0 & 0 & 0 \\
                        6 & 0 & 0 & 0 \\
                        8 & 0 & 0 & 0 \\
                        5 & 0 & 0 & 0
                    \end{bmatrix} \\
                    BA =
                    \begin{bmatrix}
                        4 & 2 & 1 & -1 \\
                        4 & 2 & 1 & -1 \\
                        0 & 0 & 0 & 0 \\
                        0 & 0 & 0 & 0
                    \end{bmatrix}
                    $
                \item 
                    $
                    AB =
                    \begin{bmatrix}
                        0 & 3 & 0 & 4 \\
                        0 & 4 & 0 & 1 \\
                        0 & 4 & 0 & 4 \\
                        0 & -6 & 0 & -1
                    \end{bmatrix} \\
                    BA =
                    \begin{bmatrix}
                        0 & 11 & 0 & -2 \\
                        0 & 0 & 0 & 0 \\
                        0 & 0 & 0 & 0 \\
                        1 & 5 & -2 & 3
                    \end{bmatrix}
                    $
                \item 
                    $
                    AB =
                    \begin{bmatrix}
                        0 & 3 & 0 & 8 \\
                        0 & -9 & 0 & 2 \\
                        0 & 0 & 0 & 8 \\
                        0 & 15 & 0 & -2
                    \end{bmatrix} \\
                    BA =
                    \begin{bmatrix}
                        -2 & 12 & 4 & -10 \\
                        0 & 0 & 0 & 0 \\
                        0 & 0 & 0 & 0 \\
                        -3 & -15 & 6 & -9
                    \end{bmatrix}
                    $
                \item 
                    $
                    AB =
                    \begin{bmatrix}
                        -4 & 4 & 2 & -3 \\
                        -1 & 10 & -4 & 9 \\
                        -4 & 8 & 8 & 0 \\
                        1 & 12 & 4 & -15
                    \end{bmatrix} \\
                    BA =
                    \begin{bmatrix}
                        -4 & -2 & -1 & 1 \\
                        2 & 10 & -4 & 6 \\
                        8 & 8 & 8 & 0 \\
                        -3 & 18 & 6 & -15
                    \end{bmatrix}
                    $
            \end{enumerate}
        \item 
            \begin{enumerate}[(i)]
                \item Invalid.
                \item Invalid.
                \item 1 $\times$ 2
                \item 2 $\times$ 1
                \item 1 $\times$ 1
                \item Invalid.
            \end{enumerate}
        \item 
    \end{enumerate}
\end{document}
